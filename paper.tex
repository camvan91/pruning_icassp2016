\documentclass{article}
\usepackage{spconf,amsmath,graphicx}
\usepackage{amssymb}
\usepackage{enumitem}

\title{Pruning Subsequence Search with Attention-Based Embedding}

\name{Colin Raffel and Daniel P. W. Ellis}
\address{LabROSA\\
    Columbia University\\
    New York, NY}

\begin{document}

\maketitle

\begin{abstract}
Finding the most similar sequence to a query in a database of sequences can be prohibitively expensive when the database size is large.
Typically, the scalability of different approaches are dependent on various ``pruning'' techniques, which use heuristics to avoid expensive comparisons against a large subset of the database.
We present an approximate pruning technique which involves embedding sequences in a Euclidean space.
Sequences are embedded using a convolutional network with a form of attention which integrates over time, which is trained using matching and non-matching pairs of sequences.
By using fixed-length embeddings, our pruning method effectively runs in constant time, making it many orders of magnitude faster when compared to full Dynamic Time Warping-based matching.
We demonstrate the effectiveness of our approach on a large-scale musical score-to-audio recording retrieval task.
\end{abstract}

\begin{keywords}
Sequence Retrieval, Attention-Based Models, Convolutional Networks, Dynamic Time Warping, Pruning
\end{keywords}

\section{Introduction}
\label{sec:intro}

Sequences\footnote{In this work, we refer to ``signals'' by the slightly more generic term ``sequences'', but the two can be used interchangeably.} are a natural way of representing data in a wide range of fields, including multimedia, environmental sciences, natural language processing, and biology.
The proliferation of sensors and decreasing cost of storage in recent years has resulted in the creation of very large databases of sequences.
Given such a database, a fundamental task is to find the database entry which is most similar to a query.
However, this task can be prohibitively expensive depending on both the method used to compare sequences and the size of the database.

An effective way to compare sequences is the Dynamic Time Warping (DTW) distance, first proposed in the context of speech utterances \cite{sakoe1978dynamic}.
DTW first computes an optimal alignment of the two sequences of feature vectors being compared, and then reports their distance as the total distance among aligned feature vectors.
The alignment step makes DTW much more robust to phase or offset errors than typical distance metrics, such as the Euclidean distance \cite{rakthanmanon2012searching}.
DTW also naturally extends to the setting where subsequence matching is allowed.
Using dynamic programming, the optimal DTW alignment can be found in time quadratic in the length of the sequences.
Despite this optimization, naive application of DTW to nearest-neighbor retrieval of the most similar sequence in a database can be prohibitively expensive.
Other dynamic programming-based ``edit distance'' metrics are used when appropriate, and are similar in both their effectiveness and inefficiency.

To mitigate this issue, a variety of ``pruning methods'' have been proposed.
The idea behind these techniques is to use heuristics to avoid computing the full DTW alignment for a large proportion of the database.
\cite{rakthanmanon2012searching} gives a broad overview of different pruning methods, and shows that their application makes exact retrieval feasible in extremely large (e.g.\ trillions of sequences) databases.
However, these methods rely on various assumptions, including that the query sequence is a subsequence of its correct match and the length of the alignment path is known a priori.

An alternative approach is to replace the DTW search with a surrogate problem which is faster to solve.
For example, \cite{papapetrou2011embedding} proposes a technique which maps sequences to a fixed-length embedding, which avoids DTW calculation when matching the query to the database.
The embedding is constructed by pre-computing the DTW distance between each sequence in the database and a small collection of ``reference'' sequences, which are chosen to optimize retrieval accuracy.
Then, to match a query sequence to the database, its DTW distance is computed against the same collection of reference sequences, and the resulting vector of DTW distances is matched against the database of embeddings using a standard Euclidean distance-based nearest-neighbor search.
The resulting algorithm is approximate, but provided state-of-the-art rsults both in terms of accuracy and speed.
This technique relies on the same assumptions as the pruning methods described in \cite{rakthanmanon2012searching}.
In addition, it presents a trade-off between efficiency and accuracy where more reference sequences generally improve accuracy but require more full DTW calculations to be made, which can be prohibitively expensive for very long or very high-dimensional sequences.

Motivated by the above issues, we propose a learning-based system for producing sequence embeddings for approximate matching.
Our approach is similar in spirit to \cite{papapetrou2011embedding}, except that it is fully general, i.e.\ it does not rely on any assumptions about the alignment length or whether the query is a subsequence of its match.
This is thanks to the fact that we utilize an attention-based neural network model which can adapt to any problem setting according to the training data provided.
Our approach is also constant-time: By mapping sequences to a fixed-length embedding, comparing a query to each database entry is a single Euclidean distance calculation, which does not become less efficient for longer or higher-dimensional sequences.

In the following section, we discuss embedding and attention-based neural network models, which we use to motivate our proposed model in Section \ref{sec:model}.
In Section \ref{sec:experiment}, we evaluate the effectiveness of our proposed model on the task of matching musical transcriptions to audio recordings of their corresponding songs in a large database.
Finally, we discuss possibilities for improvement in Section \ref{sec:discussion}.

\section{Embedding and Attention}
\label{sec:embedding}

The idea of ``embedding'' is an attractive one: Given data which is provided in a representation which is either inefficient or does not readily reveal factors of interest, can we produce a mapping to fixed-length vectors for which our data has the desired properties?
Because many simple machine learning methods are very effective given data where Euclidean distance corresponds to some notion of similarity, embedding can make data more amenable to learning and comparison.

As a concrete example, in \cite{schroff2015facenet}, a neural network architecture is used to embed images of faces such that images of the same person have small pairwise distances in the embedded space images of different people have large distances.
The authors were able to achieve state-of-the-art results on the ``labeled faces the wild'' dataset by simply thresholding the resulting distances to denote ``same person'' or ``different person''.
In addition, the embedded representation produced qualitatively useful results when performing unsupervised clustering using $k$-means.
Another highly effective application of embedding is word2vec, which is a family of models for mapping words to a Euclidean space which has desireable properties \cite{mikolov2013distributed}.
For example, the difference between ``China'' and ``Beijing'' in the embedded space is roughly equivalent to the difference between ``Rome'' and ``Italy''.

The idea of embedding has also been applied to sequences.
Notably, in \cite{sutskever2014sequence}, a recurrent network was trained to map a sequence of words in a source language to a fixed-length vector, which was then used to generate a sequence of words in a target language, resulting in a quality translation of the original sentence.
The resulting source-sentence embeddings encoded high-level linguistic characteristics, such as invariance to word order in sentences with the same meaning.
A benefit of using recurrent networks is their ability to summarize an entire sequence of arbitrary length as a fixed-length vector, which brings the possibility of learning embeddings to sequences.

One issue with using recurrent networks is their difficulty in modeling long-term dependencies \cite{bengio1994learning}.
In the embedding setting, this may result in the end of the sequence having more of an effect on the embedding than the beginning.
A recently proposed technique for mitigating this issue is ``attention'' \cite{bahdanau2014neural}.
Conceptually, an attention-based model includes a mechanism which learns to assign a weighing at each sequence step based on the current and all previous feature vectors.
When used with recurrent networks, the addition of attention has proven effective in a wide variety of domains including speech recognition, machine translation, and image captioning \cite{cho2015describing}.

At a high level, attention-based models are similar to the common technique of summarizing a sequence of feature vectors by the mean, variance, and occasionally covariance of feature dimensions.
This simple approach has seen frequent application in the domain of music information retrieval, where it has proven to be surprisingly effective at genre classification \cite{bergstra2006aggregate, tzanetakis2002musical}, instrument classification \cite{deng2008study}, artist recognition \cite{mandel2005song}, and similarity rating prediction \cite{foster2014sequential}.
Attention-based models can be seen as a generalization of this technique where both the feature representation and a weighted average are both optimized and data-adaptive.

\section{Feed-forward Attention}
\label{sec:model}

While previous work on attention-based models has focused on recurrent networks, in the present work we will use feed-forward networks for the following reasons:
First, Recurrent networks have difficulty modelling very long-term dependencies \cite{bengio1994learning}.
In the experiment described in Section \ref{sec:experiment}, sequences may have thousands of time steps, which is prohibitively long even for sophisticated models such as long short-term memory networks \cite{hochreiter1997long}.
Second, feed-forward networks are much more efficient to train and evaluate than recurrent networks because their operations can be completely parallelized, whereas recurrent networks must evaluate each time step sequentially.
Finally, attention provides its own form of temporal modelling due to the fact that it integrates over the entire sequence.
We therefore propose a ``feed-forward'' variant of attention, which computes a weighting for each sequence step independent of all other steps.

Our simplified variant of attention can be formulated as follows:
Given a matrix which comprises a sequence of $M$ $D$-dimensional vectors $X \in \mathbb{R}^{M \times D}$, we compute weights $\alpha \in \mathbb{R}^M$ using the following transformation:
\begin{equation}
\label{eq:alpha}
\alpha = \mathrm{softmax}(\sigma(Xw + b))
\end{equation}
where
$$
\mathrm{softmax}(x)_j = \frac{e^{x_j}}{\sum_{i = 1}^M e^{x_i}}
$$
and $\sigma$ is some nonlinear function (throughout this work, we will use $\sigma = \tanh$).
$w \in \mathbb{R}^D$ and $b \in \mathbb{R}$ are parameters of the transformation, which can be learned as part of a larger feed-forward network model.
After computing $\alpha$, the weighted average of the vectors in $X$ is computed by
$$
\hat{X} = \sum_{i = 1}^M \alpha_i X_{i, :}
$$
Conceptually, this attention mechanism can be thought of as using $w$, $b$, and $\sigma$ to compute scalar ``importance'' values for each vector in $X$, which are then normalized using the softmax function and used to compute a weighted mean $\hat{X}$ over all vectors in $X$.

The main difference between this formulation and the one proposed in \cite{bahdanau2014neural, cho2015describing} is that here we are only producing a single weighting $\alpha$ for the entire sequence rather than a different $\alpha$ at each time step.
This is primarily because our goal in embedding is to produce a single vector for the entire sequence whereas previous applications of attention have mainly focused on sequence-to-sequence transduction.
As a result, Equation \ref{eq:alpha} does not contain any terms for a previous attention vector or a previous hidden state, because in the present contexts these quantities do not exist.
The main disadvantage to using attention in this way is that it ignores temporal order by computing an average over the entire sequence.
However, motivated by the surprising effectiveness of the unweighted/non-adaptive Gaussian results discussed in Section \ref{sec:embedding}, we submit that this approach will be effective and verify its utility in the following section.


\bibliographystyle{IEEEbib}
\bibliography{refs}

\end{document}
